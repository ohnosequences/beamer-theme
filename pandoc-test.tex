\documentclass[10pt,compress]{beamer}

\usetheme{ohnosequences}
\usecolortheme{ohnosequences}
\usefonttheme{ohnosequences}
\usepackage{amssymb}
\usepackage{unicode-math}
\usepackage{fontspec,xltxtra,xunicode}

\let\OldHref\href
\renewcommand{\href}[2]{\OldHref[pdfnewwindow]{#1}{{#2}}}

% \setbeamertemplate{caption}[numbered]
% \setbeamertemplate{caption label separator}{:}
% \setbeamercolor{caption name}{fg=normal text.fg}
% \usepackage{amssymb,amsmath}
% \usepackage{ifxetex,ifluatex}
% \usepackage{fixltx2e} % provides \textsubscript
% \usepackage{lmodern}
%
% \usepackage{fontspec,xltxtra,xunicode}
% \defaultfontfeatures{Mapping=tex-text}
\newcommand{\euro}{€}

% use upquote if available, for straight quotes in verbatim environments
\IfFileExists{upquote.sty}{\usepackage{upquote}}{}
% use microtype if available
% \IfFileExists{microtype.sty}{\usepackage{microtype}}{}

% verbatim and code highlighting: Solarized Light
\usepackage{color}
% SOLARIZED
\definecolor{solarized@base03}{HTML}{002B36}
\definecolor{solarized@base02}{HTML}{073642}
\definecolor{solarized@base01}{HTML}{586e75}
\definecolor{solarized@base00}{HTML}{657b83}
\definecolor{solarized@base0}{HTML}{839496}
\definecolor{solarized@base1}{HTML}{93a1a1}
\definecolor{solarized@base2}{HTML}{EEE8D5}
\definecolor{solarized@base3}{HTML}{FDF6E3}
\definecolor{solarized@yellow}{HTML}{B58900}
\definecolor{solarized@orange}{HTML}{CB4B16}
\definecolor{solarized@red}{HTML}{DC322F}
\definecolor{solarized@magenta}{HTML}{D33682}
\definecolor{solarized@violet}{HTML}{6C71C4}
\definecolor{solarized@blue}{HTML}{268BD2}
\definecolor{solarized@cyan}{HTML}{2AA198}
\definecolor{solarized@green}{HTML}{859900}

\usepackage{fancyvrb}
\newcommand{\VerbBar}{|}
\newcommand{\VERB}{\Verb[commandchars=\\\{\}]}
\DefineVerbatimEnvironment{Highlighting}{Verbatim}{fontsize=\footnotesize,commandchars=\\\{\}}

% Add ',fontsize=\small' for more characters per line
\usepackage{framed}
% this is solarized light
\definecolor{shadecolor}{RGB}{253,246,227} % solarized@base3
\newenvironment{Shaded}{\vspace{\baselineskip}\begin{shaded}}{\end{shaded}\vspace{\baselineskip}}
% colored backgrd for verb
\let\oldverbatim=\verbatim
\let\endoldverbatim=\endverbatim
\renewenvironment{verbatim}[1]{%
  \vspace{\baselineskip}
  \scriptsize
  \par\setstretch{1}
  \begin{shaded}
  \begin{oldverbatim}{#1}%
}%
{%
  \end{oldverbatim}%
  \end{shaded}
  \vspace{\baselineskip}
}

\newcommand{\KeywordTok}[1]{\textbf{\textcolor{solarized@base00}{#1}}}
\newcommand{\DataTypeTok}[1]{\textcolor{solarized@blue}{#1}}
\newcommand{\DecValTok}[1]{\textcolor{solarized@violet}{#1}}
\newcommand{\BaseNTok}[1]{\textcolor{solarized@violet}{#1}}
\newcommand{\FloatTok}[1]{\textcolor{solarized@violet}{#1}}
\newcommand{\CharTok}[1]{\textcolor{solarized@cyan}{#1}}
\newcommand{\StringTok}[1]{\textcolor{solarized@violet}{#1}}
\newcommand{\CommentTok}[1]{\textcolor{solarized@base1}{\textit{#1}}}
\newcommand{\OtherTok}[1]{\textcolor{solarized@green}{#1}}
\newcommand{\AlertTok}[1]{\textcolor{solarized@yellow}{\textbf{#1}}}
% In Scala: method calls
\newcommand{\FunctionTok}[1]{\textcolor{solarized@base1}{#1}}
\newcommand{\RegionMarkerTok}[1]{\textcolor{solarized@base1}{#1}}
\newcommand{\ErrorTok}[1]{\textcolor{solarized@red}{\textbf{#1}}}
\newcommand{\NormalTok}[1]{\textcolor{solarized@base00}{#1}}


% Comment these out if you don't want a slide with just the
% part/section/subsection/subsubsection title:
% \AtBeginPart{
%   \let\insertpartnumber\relax
%   \let\partname\relax
%   \frame{\partpage}
% }
% \AtBeginSection{
%   \let\insertsectionnumber\relax
%   \let\sectionname\relax
%   \frame{\sectionpage}
% }
% \AtBeginSubsection{
%   \let\insertsubsectionnumber\relax
%   \let\subsectionname\relax
%   \frame{\subsectionpage}
% }

% \setlength{\parindent}{0pt}
% \setlength{\parskip}{6pt plus 2pt minus 1pt}
% \setlength{\emergencystretch}{3em}  % prevent overfull lines
% \providecommand{\tightlist}{%
%   \setlength{\itemsep}{0pt}\setlength{\parskip}{0pt}}
% % \setcounter{secnumdepth}{0}
% 
\usepackage{booktabs}
\usepackage[scale=2]{ccicons}

\title{Slides with Pandoc}
\subtitle{an introduction}
\author{Eduardo Pareja-Tobes}
\date{14 March 2023}

\institute{
  \href{http://era7bioinformatics.com}{{Era7} {\color{Grey-Light}bioinformatics}} - {\color{Salmon-Dark}oh}{\color{LightAmber-Dark}no}{\color{Grey}sequences}{\color{Salmon-Dark}!}
}

\begin{document}
\maketitle


% 
\section{In the morning}\label{in-the-morning}

\begin{frame}{Getting up}

This should be a nice text. Put some \emph{emphasis!}

\begin{itemize}
\itemsep1pt\parskip0pt\parsep0pt
\item
  Turn \textbf{off} alarm
\item
  Get out of bed
\end{itemize}

\end{frame}

\begin{frame}{Breakfast}

\begin{itemize}
\itemsep1pt\parskip0pt\parsep0pt
\item
  Eat eggs
\item
  Drink coffee
\end{itemize}

\end{frame}

\section{In the evening}\label{in-the-evening}

\begin{frame}{Dinner}

\begin{itemize}
\itemsep1pt\parskip0pt\parsep0pt
\item
  Eat spaghetti
\item
  Drink wine
\end{itemize}

This should be something!

\end{frame}

\section{Styles}\label{styles}

\begin{frame}{Math}

\[
  \Delta \Delta^\dagger = 1_X
\]

Adjoints can be defined through extensions, as

\[
  Lan_G 1 = F
\]

\end{frame}

\begin{frame}[fragile]{Code I}

\begin{Shaded}
\begin{Highlighting}[]
\KeywordTok{data} \DataTypeTok{DList} \NormalTok{a }\FunctionTok{=} \DataTypeTok{DLNode} \NormalTok{(}\DataTypeTok{DList} \NormalTok{a) a (}\DataTypeTok{DList} \NormalTok{a)}

\OtherTok{takeF ::} \DataTypeTok{Integer} \OtherTok{->} \DataTypeTok{DList} \NormalTok{a }\OtherTok{->} \NormalTok{[a]}
\NormalTok{takeF }\DecValTok{0}     \NormalTok{_                   }\FunctionTok{=} \NormalTok{[]}
\NormalTok{takeF (n }\FunctionTok{+} \DecValTok{1}\NormalTok{) (}\DataTypeTok{DLNode} \NormalTok{_ x next) }\FunctionTok{=} \NormalTok{x }\FunctionTok{:} \NormalTok{(takeF n next)}

\OtherTok{takeR ::} \DataTypeTok{Show} \NormalTok{a }\OtherTok{=>} \DataTypeTok{Integer} \OtherTok{->} \DataTypeTok{DList} \NormalTok{a }\OtherTok{->} \NormalTok{[a]}
\NormalTok{takeR }\DecValTok{0}     \NormalTok{_                   }\FunctionTok{=} \NormalTok{[]}
\NormalTok{takeR (n }\FunctionTok{+} \DecValTok{1}\NormalTok{) (}\DataTypeTok{DLNode} \NormalTok{prev x _) }\FunctionTok{=} \NormalTok{x }\FunctionTok{:} \NormalTok{(takeR n prev)}
\end{Highlighting}
\end{Shaded}

\end{frame}

\begin{frame}[fragile]{Code II}

Scala syntax highlighting is awful

\begin{Shaded}
\begin{Highlighting}[]
\CommentTok{// Scala has objects}
\KeywordTok{case} \KeywordTok{class} \NormalTok{test \{}
  \KeywordTok{lazy} \KeywordTok{val} \NormalTok{x = }\DecValTok{2}
  \KeywordTok{type} \NormalTok{T = Int}
  \KeywordTok{def} \NormalTok{oh: String = }\StringTok{"Buh!"}
  \KeywordTok{def} \NormalTok{uh[X <: Any](l: Boolean): X}

  \KeywordTok{val} \NormalTok{f: String => Int = _.}\FunctionTok{length}

  \KeywordTok{val} \NormalTok{g: String => Int = x => }\FunctionTok{f}\NormalTok{(x)}
\NormalTok{\}}
\end{Highlighting}
\end{Shaded}

\end{frame}

\end{document}
